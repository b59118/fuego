\section{Boards configuration}
In this document we will use such notions as \textit{targets} and \textit{boards}. Here is what they mean:
\begin{description}
\item[Target or Node] denotes a front-end Jenkins entity. Jenkins jobs are run on targets.
\item[Board] denotes a back-end entity, such as a physical board (specifically, the board or device to run tests on).
\end{description}

Board configuration is stored in \texttt{FUEGO\_RO/boards/<boardname>.board},
where \texttt{<boardname>} is the respective name of the target.

\subsection{Adding a target in Jenkins interface}
\label{sec:target-add}
The simplest method of adding a new target is to use the ftc add-nodes tool.

\subsection{Writing the board config overlay}
\label{sec:board_config}
Board config file is an overlay (See \ref{subsec:overlay_fmt}) that must inherit \texttt{base-board} and include \texttt{base-params} base classes (in that order).

The following is the step-by-step description of all mandatory environment variables that should be set by this file:

\begin{description}
\item[\texttt{TRANSPORT}:] defines how Fuego should communicate with the board.
  Currently only \texttt{ssh} is supported;
\item[\texttt{IPADDR}:]  IP address or host name of board;
\item[\texttt{SSH\_PORT}:]  ssh port number of board;
\item[\texttt{LOGIN}:]  user name for ssh login;
\item[\texttt{PASSWORD}:] password for ssh login;
\item[\texttt{FUEGO\_HOME}:] path to the directory on device the tests will run from;
\item[\texttt{PLATFORM}:] architecture of the board.
  Currently \texttt{ia32}, \texttt{arm} and \texttt{mips} are supported. Used by some of tests during compilation.
\end{description}

The following variables specify devices and mount points that are used by some
file system tests: \texttt{SATA\_DEV, SATA\_MP, USB\_DEV, USB\_MP, MMC\_DEV, MMC\_MP}.




%%% Local Variables:
%%% mode: latex
%%% TeX-master: "fuego-guide"
%%% End:
